
\chapter{绪论}
\label{c:01_introduction}


\section{选题背景及研究意义}

痴呆症是一种慢性、进展性脑部疾病,严重影响患者的认知功能、记忆力和日常生活能力。随着全球老龄化的加剧,痴呆症已成为公共健康领域的重大问题。据世界卫生组织(WHO)数据,预计到2050年,全球痴呆症患者将超过1.5亿人。这一现象不仅给患者及其家庭带来了沉重的经济和情感负担,也对社会医疗体系形成了巨大的挑战。

传统的痴呆症护理通常依赖于人工照护和医疗资源,然而面对日益增长的患者数量,这种方式已显现出人力不足、护理效率低下等问题。近年来,随着人工智能(AI)技术的快速发展,语音交互技术开始在医疗护理中展现出其潜力。智能对话系统,作为语音交互AI的重要组成部分,能够通过自然语言处理与患者进行无缝沟通,提供个性化的支持服务。研究表明,语音交互AI不仅能够减轻护理人员的负担,还能够通过情感识别、认知提醒等功能辅助痴呆症患者的日常生活。

尽管如此,语音交互AI在痴呆症护理中的应用仍面临诸多挑战。患者的认知能力下降导致了交流的复杂性,如何使AI系统更好地适应不同阶段的痴呆症患者需求,成为本领域亟待解决的关键问题。因此,本文旨在研究语音交互AI在痴呆症护理中的适应性,分析其应用潜力与局限性,并为未来的发展方向提供建议。\cite{hochr1997longshort, kapla2020scalingla}


\section{国内外研究现状}

\subsection{国内研究现状}

\subsection{国外研究现状}




\section{论文组织结构}

本文的研究内容及组织架构如下: 

第\oldref{c:01_introduction}章:\nameref{c:01_introduction}。

第\oldref{c:02_related}章:\nameref{c:02_related}。

第\oldref{c:03_main}章:\nameref{c:03_main}。

第\oldref{c:04_experiment}章:\nameref{c:04_experiment}。

第\oldref{c:05_conclusion}章:\nameref{c:05_conclusion}。





