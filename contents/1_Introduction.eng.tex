\chapter{Introduction}
\label{c:01_introduction}

%%%%%%%%%%%%%%%%%%%%%%%%
\section{Research Background}

With the global aging population on the rise, the number of dementia patients is increasing rapidly, posing significant challenges to long-term care and medical resources. Voice interaction AI technology, especially intelligent conversational systems, is gaining attention in dementia care due to its natural and convenient communication methods. This study explores the adaptability of voice interaction AI in dementia care, focusing on its potential in personalized care, emotional recognition, and memory assistance. Through experiments and user feedback data, the effectiveness and limitations of current voice interaction AI systems are evaluated, and strategies for improving their real-world performance are proposed. The findings suggest that optimized voice interaction AI systems can significantly enhance the quality of life for dementia patients, providing more intelligent solutions for dementia care. \cite{hochr1997longshort, kapla2020scalingla}


%%%%%%%%%%%%%%%%%%%%%%%%
\section{Related Works}

\subsection{Related Works of Healthcare}

\subsection{Related Works of AI Technology}



%%%%%%%%%%%%%%%%%%%%%%%%

\section{Chapter Organization}
The research content and organizational structure of this thesis are as follows:

Chapter 1: \nameref{c:01_introduction}. This chapter ...

Chapter 2: \nameref{c:02_related}. This chapter ...

Chapter 3: \nameref{c:03_main}. This chapter ...

Chapter 4: \nameref{c:04_experiment}. This chapter ...

Chapter 5: \nameref{c:05_conclusion}. This chapter ...




